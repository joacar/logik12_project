\documentclass[a4paper,12pt]{article}
\usepackage[T1]{fontenc}
\usepackage{textcomp}
\usepackage[utf8x]{inputenc}
\usepackage[swedish,english]{babel}
\usepackage{graphicx}
\usepackage{float}
% IMPORTANT! This need to be the last package loaded since it overwrites features from others to work correctly
\usepackage[pdfusetitle=true, pdfauthor={Pascal Chatterjee and Joakim Carselind}, pdfsubject={Solving Hashiwokakero, a Japanese logic puzzle, in prolog}, pdfkeywords={logic puzzle, prolog, hashiwokakero}, pdftex]{hyperref}

\title{Solving Hashiwokakero in Prolog}
\author{Pascal Chatterjee and Joakim Carselind \\ \small{\{joacar, pascalc\}@kth.se}}

\begin{document}
\maketitle

\section*{Introduction}
Hashiwokakero is a Japanese puzzle that involves connecting islands with bridges under strict constraints:
\begin{enumerate}
	\item They must begin and end at distinct islands, travelling a straight line in between.
	\item They must not cross any other bridges or islands.
	\item They may only run perpendicularly.
	\item At most two bridges connect a pair of islands.
	\item The number of bridges connected to each island must match the number on that island.
	\item The bridges must connect the islands into a single connected group.
\end{enumerate}
The number of the islands is one to eight.
The puzzle is a typical constraint satisfaction problem and Prolog is well suited for dealing with problems of that nature.

Translate the six constraints from english into logic clauses, translate those clauses into facts that Prolog can interpretate and start the resolution process. Sounds simple?

\section*{The problem}
The constraints have certain domain in the meaning of refering to a local constraint or a global constraint. The constraint that all islands must be connected into a single connected group is a global constraint.

\section*{A first attempt}
A first attempt, a naive approach if one so like, to solve the problem is to generate all possible configurations and test each one of them until a configuration satisifying the constraints is found and accepted as the solution.

\end{document}